\documentclass[10pt]{beamer}
\usepackage[utf8]{inputenc}
\usepackage{multirow,rotating}
\usepackage{color}
\usepackage{hyperref}
\usepackage{tikz-cd}
\usepackage{array}
\usepackage{siunitx}
\usepackage{mathtools,nccmath}%
\usepackage{etoolbox, xparse} 
\usetheme{CambridgeUS}
\usecolortheme{dolphin}
% set colors
\definecolor{myNewColorA}{RGB}{139, 33, 49}
\definecolor{myNewColorB}{RGB}{139, 33, 49}
\definecolor{myNewColorC}{RGB}{139, 33, 49}
\setbeamercolor*{palette primary}{bg=myNewColorC}
\setbeamercolor*{palette secondary}{bg=myNewColorB, fg = white}
\setbeamercolor*{palette tertiary}{bg=myNewColorA, fg = white}
\setbeamercolor*{titlelike}{fg=myNewColorA}
\setbeamercolor*{title}{bg=myNewColorA, fg = white}
\setbeamercolor*{item}{fg=myNewColorA}
\setbeamercolor*{caption name}{fg=myNewColorA}
\usefonttheme{professionalfonts}
\usepackage{natbib}
\usepackage{hyperref}
\graphicspath{{../img/}}
%------------------------------------------------------------
\titlegraphic{\includegraphics[height=0.75cm]{img/Chapman University_Engineering_Logo.png}} 
\titlegraphic{%
\includegraphics[width=3.0cm]{chapdimensionsseal_220.png}
}
\setbeamerfont{title}{size=\large}
\setbeamerfont{subtitle}{size=\small}
\setbeamerfont{author}{size=\small}
\setbeamerfont{date}{size=\footnotesize}
\setbeamerfont{institute}{size=\footnotesize}
\title[CU]{Spatiotemporal Trends in Pacific Northwest Wildfire Occurrence and Firefighting Resource Demand}
\author[Blaise Rettig]{Blaise Rettig}
\institute[CU]{Chapman University, Fowler School of Engineering}
\date[\textcolor{white}{}]
{CPSC 370 Data Science for Sustainability and Environmental Analysis\\
Dec 08, 2025}
%------------------------------------------------------------
\AtBeginSection[]{
  \begin{frame}
  \vfill
  \centering
  \begin{beamercolorbox}[sep=8pt,center,shadow=true,rounded=true]{title}
    \usebeamerfont{title}\insertsectionhead\par%
  \end{beamercolorbox}
  \vfill
  \end{frame}
}
%------------------------------------------------------------
\begin{document}
\frame{\titlepage}

\begin{frame}
\frametitle{Outline}
\tableofcontents
\end{frame}

%------------------------------------------------------------
\section{Overview and Objectives}

\begin{frame}{The Pacific Northwest Wildfire Challenge}
\textbf{Growing Concerns:}
\begin{itemize}
    \item Rising temperatures and prolonged droughts
    \item Increased development at wildland-urban intersection
    \item More frequent and severe fire events
    \item Complex impacts on infrastructure and communities
\end{itemize}

\vspace{0.3cm}
\textbf{The Gap:}
\begin{itemize}
    \item Traditional research focuses on burn severity or climate drivers
    \item Few models connect occurrence to downstream consequences
    \item Limited predictive frameworks for resource allocation
\end{itemize}
\end{frame}

\begin{frame}{Research Objectives}
\textbf{Primary Goal:} Identify spatiotemporal trends and build a predictive framework for wildfire impacts

\vspace{0.3cm}
\textbf{Specific Objectives:}
\begin{enumerate}
    \item Analyze wildfire occurrence patterns (2002--2024)
    \item Model relationships between fire characteristics and environmental/infrastructural factors
    \item Develop Python-based predictive model for impact severity indices
    \item Visualize trends using QGIS for decision-making support
\end{enumerate}

\vspace{0.3cm}
\textbf{Study Area:} Oregon, Washington, Idaho (~568,000 km²)
\end{frame}

%------------------------------------------------------------
\section{Data Used and Methods}

\begin{frame}{Data Sources}
\textbf{Wildfire Activity:}
\begin{itemize}
    \item MTBS: Fire perimeters, burn severity (1984--2024)
    \item ICS-209: Resource deployment, suppression costs
\end{itemize}

\vspace{0.2cm}
\textbf{Environmental Variables:}
\begin{itemize}
    \item PRISM: Temperature and precipitation
    \item TerraClimate: Wind speed
    \item Stanford EchoLab: Wildfire smoke PM2.5
    \item U.S. Drought Monitor: Drought indices
\end{itemize}

\vspace{0.2cm}
\textbf{Infrastructure \& Socioeconomic:}
\begin{itemize}
    \item EIA: Electricity generation, power plant locations
    \item CDC/ATSDR: Social Vulnerability Index (SVI)
\end{itemize}
\end{frame}

\begin{frame}{Analytical Methods}
\textbf{Temporal Analysis (Python):}
\begin{itemize}
    \item Time series decomposition and forecasting
    \item Correlation analysis: climate anomalies vs. fire activity
    \item Libraries: pandas, geopandas, numpy, matplotlib
\end{itemize}

\vspace{0.3cm}
\textbf{Spatial Analysis (QGIS):}
\begin{itemize}
    \item Hotspot detection and kernel density estimation
    \item Raster overlay for infrastructure proximity
    \item Atmospheric difference mapping
\end{itemize}

\vspace{0.3cm}
\textbf{Predictive Modeling (scikit-learn):}
\begin{itemize}
    \item Gradient Boosting Regressors (150 estimators, lr=0.05)
    \item Outputs: Resource demand, evacuation risk, structure threat, suppression cost
    \item Log-transform for skewed distributions
\end{itemize}
\end{frame}

%------------------------------------------------------------
\section{Results}

\begin{frame}{Temporal Trends: Fire Activity Over Time}
% Leave space for Figure 1
\begin{center}
\includegraphics[width=\linewidth]{1Figure_1.png}
\end{center}

\vspace{0.2cm}
\textbf{Key Findings:}
\begin{itemize}
    \item Major peaks in 2006 and 2012: High fire count + large burned area
    \item Other peaks (2000, 2017): Many fires, but controlled burn area
    \item Suggests variable firefighting effectiveness across years
\end{itemize}
\end{frame}

\begin{frame}{Climate Relationships: Temperature and Precipitation}
\begin{columns}
\column{0.5\textwidth}
\begin{center}
\includegraphics[width=0.75\linewidth]{1Figure_2.png}
\end{center}
\begin{itemize}
    \item Weak negative correlation
    \item High variability
\end{itemize}

\column{0.5\textwidth}
\begin{center}
\includegraphics[width=0.75\linewidth]{1Figure_3.png}
\end{center}
\begin{itemize}
    \item Similar high variability
    \item Simple metrics insufficient
\end{itemize}
\end{columns}

\vspace{0.3cm}
\textbf{Drought Analysis:}
\begin{itemize}
    \item No significant temp difference during drought (p=0.437)
    \item Temperature and precipitation stronger predictors than drought indices
\end{itemize}
\end{frame}

\begin{frame}{Fire Season and Reburn Patterns}
\begin{columns}
\column{0.5\textwidth}
\begin{center}
\includegraphics[width=\linewidth]{1Figure_7.png}
\end{center}
\begin{itemize}
    \item Expanding fire season
    \item More active fire days/year
\end{itemize}

\column{0.5\textwidth}
\begin{center}
\includegraphics[width=\linewidth]{1Figure_6.png}
\end{center}
\begin{itemize}
    \item \textbf{929 reburned areas}
    \item Growing prevalence
    \item Short-interval reburns
\end{itemize}
\end{columns}
\end{frame}

\begin{frame}{Firefighting Resource Deployment}
\includegraphics[width=\linewidth]{2Figure_1.png}
\end{frame}

\begin{frame}{Firefighting Resource Deployment}
\includegraphics[width=\linewidth]{2Figure_2.png}
\vspace{0.2cm}
\textbf{Critical Findings:}
\begin{itemize}
    \item Structures threatened: 46.8 median personnel
    \item No structures: 26.1 median personnel
    \item \textbf{Nearly double} the resources for structural protection
    \item 143 fires caused evacuations
    \item 481 fires resulted in structure loss
    \item Average loss rate when threatened: 7.73\%
\end{itemize}
\end{frame}

\begin{frame}{Spatial Patterns: Fire Hotspots}
\includegraphics[width=\linewidth]{6Figure_6.png}
\end{frame}

\begin{frame}{Spatial Patterns: Fire Hotspots}
\includegraphics[width=\linewidth]{6Figure_7.png}
\end{frame}

\begin{frame}{Spatial Patterns: Fire Hotspots}

\vspace{0.2cm}
\textbf{Persistent Fire Hotspots:}
\begin{itemize}
    \item Eastern Oregon
    \item Central Washington
    \item Southern Idaho
\end{itemize}

\vspace{0.2cm}
\textbf{Characteristics:}
\begin{itemize}
    \item Dry forests, shrublands, grasslands
    \item High fuel loads
    \item Align with greatest temperature increases and precipitation declines
\end{itemize}
\end{frame}

\begin{frame}{Climate Change Impacts}
\begin{columns}
\column{0.5\textwidth}
\begin{center}
\includegraphics[width=\linewidth]{6Figure_2.png}
\end{center}
\textbf{2024 vs. 2000:}
\begin{itemize}
    \item Range: -2.91°C to +3.41°C
\end{itemize}

\column{0.5\textwidth}
\begin{center}
\includegraphics[width=\linewidth]{6Figure_3.png}
\end{center}
\textbf{2024 vs. 2000:}
\begin{itemize}
    \item Range: -584mm to +2173mm
\end{itemize}
\end{columns}

\vspace{0.3cm}
Clear spatial patterns of warming and changing precipitation align with fire hotspots
\end{frame}

\begin{frame}{Social Vulnerability}
\begin{center}
\includegraphics[width=0.6\linewidth]{6Figure_9.png}
\end{center}

\vspace{0.2cm}
\textbf{Most Vulnerable Counties:}
\begin{itemize}
    \item Oregon: Malheur, Jefferson, Umatilla
    \item Washington: Adams, Yakima, Okanogan
    \item Idaho: Washington, Elmore, Gooding
\end{itemize}

\vspace{0.2cm}
\textbf{Key Finding:} Highest vulnerability counties experience elevated fire exposure
\end{frame}

\begin{frame}{Predictive Model Performance}
\begin{center}
\includegraphics[width=\linewidth]{model_performance_improved.png}
\end{center}
\end{frame}

\begin{frame}{Predictive Model Performance}
\vspace{0.2cm}
\textbf{R² Values (Variance Explained):}
\begin{itemize}
    \item Structure Threat: 68.5\% (best performer)
    \item Suppression Cost: 25.3\%
    \item Resource Demand: 18.9\%
    \item Evacuation Risk: 6.0\% (poorest)
\end{itemize}

\vspace{0.2cm}
\textbf{Model Architecture:}
\begin{itemize}
    \item Gradient Boosting with 150 estimators
    \item Learning rate: 0.05, max depth: 6
    \item Log-transform for skewed distributions
\end{itemize}
\end{frame}

\begin{frame}{Model Predictions: Small vs. Large Fire}
\begin{center}
\includegraphics[width=\linewidth]{prediction_examples_improved.png}
\end{center}
\end{frame}

\begin{frame}{Model Predictions: Small vs. Large Fire}
\vspace{0.2cm}
\begin{columns}
\column{0.5\textwidth}
\textbf{Small Fire (100 ha):}
\begin{itemize}
    \item 14 personnel
    \item 87 evacuated
    \item 22 structures
    \item \$1.07M cost
\end{itemize}

\column{0.5\textwidth}
\textbf{Large Fire (10,000 ha):}
\begin{itemize}
    \item 538 personnel
    \item 133 evacuated
    \item 319 structures
    \item \$4.95M cost
\end{itemize}
\end{columns}

\vspace{0.3cm}
Model shows expected scaling with fire size, but high uncertainty in human decision variables
\end{frame}

%------------------------------------------------------------
\section{Highlights of Study}

\begin{frame}{Key Findings}
\textbf{Clear Spatiotemporal Trends:}
\begin{itemize}
    \item Increasing frequency and burned area (1984--2024)
    \item Expanding fire season length
    \item 929 reburned areas identified
    \item Persistent hotspots in eastern PNW
\end{itemize}

\vspace{0.3cm}
\textbf{Infrastructure \& Social Impacts:}
\begin{itemize}
    \item Double personnel for structure protection
    \item Spatial correlation with energy infrastructure
    \item Significant disparities in vulnerability across communities
    \item PM2.5 exposure increased during major fire years
\end{itemize}
\end{frame}

\begin{frame}{Model Limitations and Insights}
\textbf{Why the Model Struggles:}
\begin{itemize}
    \item Operational decisions involve real-time factors beyond historical data
    \item Political pressures and resource constraints not captured
    \item Evacuation orders depend on local protocols, not just fire characteristics
    \item High cost variability from jurisdictional differences
    \item I'm not great at this
\end{itemize}

\vspace{0.3cm}
\textbf{What Works:}
\begin{itemize}
    \item Structure threat prediction (68.5\% R²)
    \item Spatial hotspot identification for resource pre-positioning
    \item Social vulnerability mapping for preparedness planning
\end{itemize}
\end{frame}

\begin{frame}{Implications for Management}
\textbf{Practical Applications:}
\begin{itemize}
    \item Spatial hotspot analysis → firefighting asset pre-positioning
    \item Social vulnerability mapping → targeted preparedness programs
    \item Climate trend analysis → long-term resource planning
    \item Structure threat model → evacuation planning support
\end{itemize}

\vspace{0.3cm}
\textbf{Future Research Directions:}
\begin{itemize}
    \item Incorporate high-resolution fire weather indices
    \item Add real-time fuel and/or moisture measurements
    \item Include suppression strategy data
    \item Extend to post-fire recovery and economic impacts
    \item Explore neural networks for non-linear relationships
\end{itemize}
\end{frame}

\begin{frame}{Conclusions}
\textbf{This research demonstrates:}
\begin{itemize}
    \item Pacific Northwest wildfires show clear increasing trends
    \item Climate variables have complex relationships with fire activity
    \item Significant disparities exist in impact distribution
    \item Predicting human dimensions (evacuation, deployment) remains challenging
\end{itemize}

\vspace{0.3cm}
\textbf{Central Message:}
\begin{center}
\textit{Effective wildfire management must integrate climate data, spatial analysis, and social considerations to protect communities and ecosystems.}
\end{center}

\vspace{0.3cm}
\textbf{Repository:} \href{https://github.com/blaiserettig/predicting-fire}{github.com/blaiserettig/predicting-fire}
\end{frame}

%------------------------------------------------------------
\section*{Acknowledgement}  
\begin{frame}
\textcolor{myNewColorA}{\huge{\centerline{Thank you!}}}
\vspace*{0.5cm}
\textcolor{myNewColorA}{\Large{\centerline{brettig@chapman.edu}}}
\end{frame}

\end{document}